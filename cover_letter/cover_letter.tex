\documentclass[10pt,a4paper]{article}
\usepackage[utf8]{inputenc}
\usepackage[english]{babel}
\usepackage{amsmath}
\usepackage{amsfonts}
\usepackage{amssymb}
\usepackage[left=3cm,right=3cm,top=3cm,bottom=3cm]{geometry}
\begin{document}
\noindent
To whom it may concern:
\medskip
\\
We would like to explain our intention to publish the results of our recent research on PLOS Computational Biology. The field of studies is computational neuroscience, more specifically the theoretical study of the formation, organization and dynamics of recurrent cortical networks. This study was particularly focused on the concept of \textit{diffusive homeostasis} and its interplay with other, well-established mechanisms of cortical plasticity.

The statistics of neural firing rates in cortical networks are determined by statistical properties of the input each neuron receives by means of its afferent synaptic connections, as well as cell-intrinsic biophysical parameters. Both of these components are subject to change, and the study of these dynamics have revealed many different biological mechanisms, acting at different places and on different timescales. Although individual mechanisms of synaptic and intrinsic plasticity have been analyzed and modeled to a great extent, combining these building blocks into a realistic model of cortical activity and self-organization remains a challenging task. Particularly, it is not clear how these basal processes give rise to statistics of neuronal firing rates, which are known to follow heavy-tailed, log-normal distributions.

In this paper, we present a model of cortical self-organization that exhibits key features of network topology and neural activity known to be present in cortical networks. It utilizes well-established, biologically plausible principles of synaptic plasticity in combination with a diffusive homeostatic mechanism. It has been shown in previous theoretic studies that the latter can give rise to log-normal distributions of neural firing rates when applied to recurrent networks. However, our model is the first to show that diffusive homeostasis works in combination with other well-known mechanisms of synaptic plasticity, giving rise to the aforementioned features of cortical wiring, including a log-normal distribution of synaptic weights and an over-representation of bidirectional connectivities.
In addition, we present a theoretical analysis of the diffusive homeostatic mechanism, leading to the hypothesis that fluctuations of local neuron densities may determine neural firing rates.

We believe that the results of our study suit the profile of PLOS Computational Biology due to its computational and theoretical background applied to a biophysical topic that is the subject of current ongoing research. Readers of PLOS Computation Biology focusing on theoretical and computational neuroscience may use our model as an inspiration for further research, addressing more specific questions, e.g. regarding learning in recurrent networks. Researchers with access to experimental data of neural activity in cortical networks might feel inclined to test our predictions regarding the influence of neuron density onto firing rates, and we strongly encourage this.

We hope that our explanations have raised your interest and that you consider our work for publication.
\medskip
\\
Sincerely,

Fabian Schubert
\end{document}